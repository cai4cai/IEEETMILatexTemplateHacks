% For some reason ieeecolor presents itself as IEEEtran
% Silence the warning
\RequirePackage{silence}
\WarningFilter{latex}{You have requested document class `ieeecolor'}

\documentclass[journal,twoside,web]{ieeecolor}
\usepackage{tmi}
\usepackage{graphicx}

% Useful for lorem ipsum purposes only
\usepackage{mwe}

% Fix ieeecolor's \caption
% See https://tex.stackexchange.com/a/583096/172400
\usepackage{etoolbox}
\makeatletter
\@ifundefined{color@begingroup}%
  {\let\color@begingroup\relax
   \let\color@endgroup\relax}{}%
\def\fix@ieeecolor@hbox#1{%
  \hbox{\color@begingroup#1\color@endgroup}}
\patchcmd\@makecaption{\hbox}{\fix@ieeecolor@hbox}{}{\FAILED}
\patchcmd\@makecaption{\hbox}{\fix@ieeecolor@hbox}{}{\FAILED}
%

% Allow sufigures with the subcaption package
% See https://www.michaelshell.org/tex/ieeetran/#:~:text=subcaption
\WarningFilter{caption}{Unknown document class (or package)}

\makeatletter
\let\MYcaption\@makecaption
\makeatother
\usepackage[font=footnotesize]{subcaption}
\makeatletter
\let\@makecaption\MYcaption
\makeatother
%

% Define some handy commands
\newcommand{\figref}[1]{Fig.~\ref{#1}}
\newcommand{\tabref}[1]{Table~\ref{#1}}
\newcommand{\secref}[1]{Section~\ref{#1}}
%

% Not sure why this is needed but it is in the official template
\def\BibTeX{{\rm B\kern-.05em{\sc i\kern-.025em b}\kern-.08em
    T\kern-.1667em\lower.7ex\hbox{E}\kern-.125emX}}
%

% Define the running headings
\markboth{Submitted to \journalname}
{T. Vercauteren \MakeLowercase{\textit{et al.}}: IEEE TMI Template Hacks}
%

% Load hyperref last unless you know better
% As mentioned in ieeecolor.cls a patch is needed for citations
\makeatletter
\let\NAT@parse\undefined
\makeatother
\usepackage[colorlinks=true,allcolors=black]{hyperref}
%

\begin{document}
% Add some further control on the bibliography
\bstctlcite{IEEEexample:BSTcontrol}

\title{IEEE TMI Template Hacks}
\author{Tom Vercauteren
\thanks{This work was supported by some funder.}%
\thanks{T. Vercauteren is with King's College London (e-mail:\href{mailto:tom.vercauteren@kcl.ac.uk}{tom.vercauteren@kcl.ac.uk}). }%
}


\maketitle

\begin{abstract}
\blindtext
\end{abstract}

\begin{IEEEkeywords}
Keyword 1, keyword 2.
\end{IEEEkeywords}

\section{Introduction}
\blindtext[2]

\section{Method}
\blindmathpaper

\begin{figure}[tbh!]
\centering
\includegraphics[width=0.8\linewidth]{example-image}
\caption{Example Image.\label{fig:example}}
\end{figure}

\blindtext
Add some references~\cite{article-full,inproceedings-full,phdthesis-full,Fidon:PAMI:2024,Bakas:arXiv:2018,Alabi:MedIA:2025} and a link to \figref{fig:example}.

\section{Conclusion}
\blindtext[3]

% References
\bibliographystyle{IEEEtran}
\bibliography{%
  IEEEabrv.bib,% IEEE journal names as provided by the IEEEtran package
  jn_abbrv,% Other journal names using NLM Title Abbreviation
  xampl,% For lorem ipsum only
  references,% The actual references for this manuscript
  ieeebib_settings% Settings, needs to be last
  }
 

\end{document}
